\section{JSON WEB TOKEN}
\subsection{Тайлбар}
JSON Web Token (JWT) нь нээлттэй стандарт (RFC 7519) бөгөөд мэдээллийг JSON объект болгон талуудын хооронд найдвартай дамжуулах шалгарсан арга юм. Дамжуулагдах мэдээлэл нь server дээр гарын үсгээр (Signature) баталгаажсан байх тул итгэж болохуйц юм. JWT-г баталгаажуулахдаа \textbf{hash} хийх hmac алгоритм эсвэл нээлттэй хаалттай (public/private) түлхүүрийн хослолыг ашиглаж болдог (RSA эсвэл ECDSA).

\subsection{Бүрэлдэхүүн}
\begin{itemize}
	\item \textbf{Толгой хэсэг}: Толгой хэсэг нь ихэвчлэн хоёр хэсгээс бүрдэнэ: токенын төрөл (JWT) ба HMAC SHA256 эсвэл RSA зэрэг ашиглаж буй гарын үсэг зурах алгоритм.
	\item \textbf{Их бие (payload)}: Их бие нь токенын хэрэглэгчийн мэдээллийг агуулдаг. Хэрэглэгчийн мэдээллийг хадгалахын тулд хэрэглэгчийн ID, хэрэглэгчийн нэр, хэрэглэгчийн эрхийг хадгалдаг. Гэхдээ энэ хэсэгт нууц мэдээлэл хадгалж болохгүй, нэмэлт мэдээлэл л хадгална, яагаад гэвэл энэ нь хэн ч харж болох base64 string байдаг.
	\item \textbf{Гарын үсэг (signature)}: Гарын үсгийн хэсгийг үүсгэхийн тулд их бие, толгой хоёрийг авч толгой хэсэгт заасан алгоритмыг ашиглан гарын үсэг зурна. Гарын үсэг зурна гэдэг нь ерөнхийдөө энэхүү мэдээллийг ХАЙШ хийх ба дараа нь үүссэн ХАЙШ нь үүнийг баталгаажуулахад ашиглагдах юм.
\end{itemize}

\begin{lstlisting}[language=Python,caption={JWT олгох код},frame=single]
	def _create_token(token_type: str, lifetime: timedelta, sub: str) -> str:
	payload = {}
	expire = datetime.utcnow() + lifetime
	payload["type"] = token_type

	payload["exp"] = expire
	payload["iat"] = datetime.utcnow()
	payload["sub"] = str(sub)

	return jwt.encode(payload, settings.JWT_SECRET, algorithm=settings.ALGORITHM)

\end{lstlisting}

\subsubsection{JWT токен шалгах}
Хэрэглэгч Бакэндтэй харьцах болгондоо өөрийг токенийг илгээж жинхэнэ хэрэглэгч гэдгээ баталгаажуулах ёстой. Тийм учраас JWT үүсгэхдээ анхаарах хамгийн чухал зүйлс бол токен олгосон цаг, токен дуусах цаг хоёр юм.
\begin{lstlisting}[language=Python,caption={JWT тайлж баталгаажуулах код},frame=single]
	async def get_current_user(
    db: Session = Depends(get_db),
    token: str = Depends(oauth2_scheme),
) -> User:
    try:
        payload = jwt.decode(
            token,
            settings.JWT_SECRET,
            algorithms=[settings.ALGORITHM],
            options={"verify_aud": False},
        )
        sub: str = payload.get("sub")
        if sub is None:
            raise credentials_exception
        id, role = sub.split("|")
        token_data = TokenData(username=id, role=role)
    except JWTError:
        raise credentials_exception

    if token_data.role not in ["admin", "OSPdesigner", "OSPsalesman", "client"]:
        raise credentials_exception

    user = db.query(User).filter(User.id == token_data.username).first()
    if user is None:
        raise credentials_exception

    return user

\end{lstlisting}