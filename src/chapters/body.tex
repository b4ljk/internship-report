\section{Pillow сан ашиглан динамик байдлаар текстээс зураг үүсгэх}
\subsection{Database дээр зурагний мэдээлэл хадгалах table нэмэх}
Database-н table дээр өөрчлөлт оруулахдаа бид sqlalchemy ашиглаж байгаа ба доор бичсэн моделийн дагуу бид migrate хийх юм. Үүний тулд back-end ажиллаж байгаа Docker container-лүү shell нээж төслийн root хэсгээс
\begin{lstlisting}[language=bash]
	alembic revision --autogenerate -m "your commit message"
\end{lstlisting}
гэсэн коммандын ашиглан өөрлөлт оруулах мэдээллиг үүсгэн. Дараа нь
\begin{lstlisting}[language=bash]
	alembic upgrade head
\end{lstlisting}
комманд хийснээр Датабазын модел бүрэн өөрчлөгдөнө.

\begin{lstlisting}[language=Python,caption={Table-рүү оруулсан өөрчлөлт},frame=single]
	class User(Base):
		# ...
		# Other table information
		text_title_string = Column(String(255), nullable=False, default="")
		text_title_color = Column(String(255), nullable=False, default="#000000")
		text_title_font_size = Column(Integer, nullable=False, default=0)
		text_title_font_name = Column(String(255), nullable=False, default="")
		text_title_is_vertical = Column(Integer, nullable=False, default=0)
	\end{lstlisting}
\end{document}
